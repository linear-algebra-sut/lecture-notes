\chapter{فضاهای خطی}

\section{فضا‌های خطی}
فضا‌ی $R^n$ از همه‌ی بردار‌های ستونی $n$ مؤلفه‌ای حقیقی تشکیل شده است که در آن، حاصل‌جمع هر دو بردار و حاصل‌ضرب آن‌ها در اسکالر، کماکان عضوی از $R^n$ است؛ به عبارت دیگر، هر ترکیب خطی از بردارها، در فضا قرار داشته باشد. در این شرایط،  $R^n$ را یک فضا‌ی خطی می‌نامیم.\\
علاوه بر فضاهای با بعد محدود، فضا‌ی بی‌نهایت بعدی $R^\infty$ نیز وجود دارد که بردار‌های درون آن به صورت $x = (x_1,x_2,\cdots)$ هستند. هر بردار درون آن، بی‌نهایت مؤلفه دارد و به ازای هر دو بردار $x$ و $y$ و هر اسکالر مانند $c$، بردارهای $x+y$ و $cx$ نیز درون فضا هستند، یعنی:
$$x,y\in R^\infty, c\in R \qquad\Rightarrow \qquad x+y\in R^\infty, cx\in R^\infty$$
پس $R^\infty$ هم یک فضای خطی است. برای مثال، فضا‌ی توابع را در نظر بگیرید و فرض کنید $V = \{f:[0,1]\rightarrow{}R\}$. در این صورت، اگر در نظر بگیریم $g,f\in V$  و $c\in R$، می‌توان به راحتی دید که $f+g\in R$ و $cf\in R$.\\ فضا‌ی چند‌جمله‌ای‌ها از درجه‌ی حداکثر $n$ نیز خواص مشابهی دارد.\\
\textbf{تعریف فضا‌ی خطی:}
مجموعه‌ای مانند $V$ به همراه جمع و ضرب اسکالر روی $V$، $(V,+,\cdot)$، فضا‌ی خطی است، اگر و تنها اگر ویژگی‌های زیر درست باشند:\\
\begin{enumerate}
	\item جا‌به‌جایی:\quad به ازای هر $u$ و $v$ \qquad $u+v = v+u$
	\item شرکت‌پذیری:\quad به ازای هر $u,v,w\in V$ و هر $a,b\in R$:
	$$(u+v)+w = u+(v+w)$$
	$$(ab)v=a(bv)$$
	\item عضو خنثی جمعی:  عضوی مانند $0$ در $V$ وجود داشته باشد به طوری که به ازای هر $v\in V$ داشته باشیم $v+0=v$.
	\item وارون جمعی: به ازای هر $v\in V$ عضوی مانند $w\in V$ باشد به طوری که $v+w=0$.
	\item عضو خنثی ضربی: عضوی مانند $1$ در $V$ وجود داشته باشد به طوری که به ازای هر $v\in V$ داشته باشیم $1\cdot v = v$.
	\item توزیع‌پذیری: به ازای هر $u,v\in V$ و $a,b\in R$ :
	$$(a+b)v = av + bv$$
	$$a(u+v) = au+av$$
\end{enumerate}
توجه کنید که فضا‌ی خطی در تعریف فوق، فضا‌ی خطی روی اعداد حقیقی یا همان «فضا‌ی خطی حقیقی» است. می‌توان به طور مشابه، فضا‌ی خطی را روی اعداد مختلط $C$ نیز تعریف کرد که آن را فضا‌ی خطی \textbf{مختلط} می‌نامند. برای تشکیل چنین فضایی، کافی است که اسکالر‌ها را از $C$ انتخاب کنیم.\\
\textbf{مثال:}
فرض کنید $A\in M_{n}(R)$ باشد. ستون‌های $A$ را با $A_{1}\cdots A_{n}$ نمایش می‌دهیم. قرار دهید:
\begin{flushleft}$V = \{c_{1}A_{1}+\cdots+c_{n}A_{n} \; | \; c_{i}\in R, 1\leq i \leq n\}$\end{flushleft}
به راحتی می‌توان دید که $V$ یک فضا‌ی خطی است. فضا‌ی خطی $V$ را فضا‌ی \textbf{ستونی} ماتریس $A$ می‌نامیم و با $C(A)$ نمایش می‌دهیم.\\
\textbf{تعریف زیرفضای خطی:}
فرض کنید $V$ یک فضا‌ی خطی باشد. زیر‌فضا‌ی خطی، زیر‌مجموعه‌ای ناتهی از $V$ مانند $W$ است، به طوری که در شرایط فضا‌ی خطی صدق کند، یعنی:
\begin{enumerate}
	\item اگر هر بردار $x$ و $y$ از زیر‌فضا‌ی $W$ را با هم جمع کنیم ، $x+y$ در $W$ باشد.
	\item اگر $v$ بردار دلخواهی در $W$ و $c$ اسکالری دلخواه باشد $cv$ در $W$ باشد.
\end{enumerate}

\textbf{یادداشت:} در مثال قبل $C(A)\subseteq R^m$ و $C(A)$ فضا‌ی خطی است. بنابراین انتظار داریم $C(A)$ به عنوان زیر‌مجموعه‌ای از $R^m$، یک زیر‌فضا‌ی خطی از $R^m$ باشد.\\\\

\textbf{نکته:}
اگر $V$ فضا‌ی خطی و $W\subseteq V$ زیر‌فضا‌ی خطی باشد آنگاه $0\in W$.\\\\
\textbf{مثال:}
فرض کنید $S_{1} = \{(x,y)\in R^2|x\geq0 , y\geq0\}$. $S_{1}$ زیر‌فضا نیست، زیرا: $$v = \begin{bmatrix}
1\\1
\end{bmatrix}, c=-1 \quad\Rightarrow\quad c\cdot v\notin S_{1}$$
حال به مجموعه‌ی $S_{1}$ ربع سوم صفحه‌ی مختصات را نیز اضافه می‌کنیم، یعنی: $${S_{2}= \{(x,y)\in R^2|x\geq0,y\geq0 \quad\textit{or}\quad x<0 , y<0\} }$$ در این صورت:
$$v_{1}=\begin{bmatrix}
1\\2
\end{bmatrix}, v_{2}=\begin{bmatrix}
-2\\-1
\end{bmatrix}\quad\Rightarrow\quad v_{1}+v_{2} = \begin{bmatrix}
-1\\1
\end{bmatrix}\notin S_{2}$$
پس $S_{2}$ نیز زیر‌فضا نیست.\\
\textbf{توجه: }کوچک‌ترین زیر‌فضا‌ی شامل $S_{1}$، خود $R^2$ است. همچنین $S_{1}$ زیر‌مجموعه‌ای از $R^2$ است که تحت جمع بسته بوده، ولی تحت ضرب اسکالر بسته نیست.\\\\
\textbf{مثال:}
کوچک‌ترین زیر‌فضا‌ی $M_{n}(R)$ که شامل همه‌ی ماتریس‌های متقارن و همه‌ی ماتریس‌های مثلثی است را بیابید.

\textbf{حل:}
$$W=\{\alpha S+ \beta L | S \text{متقارن}, L\text{پایین‌مثلثی}, \alpha,\beta\in R\}$$
فرض کنید $A\in M_{n}(R)$. در این صورت قرار دهید:
$$S_{ij} = \Bigg\{\begin{array}{ccccc}
A_{ij} &  i<j\\
0 & i=j\\
A_{ji} & i>j
\end{array} \quad\Rightarrow\quad (A-S)_{ij} = \Bigg\{\begin{array}{ccc}
0 & i\leq j\\
a_{ij} - a_{ji}&i>j
\end{array}$$
بنابراین $A = (A-S) + S$ که در آن $A-S$ پایین‌مثلثی و $S$ متقارن است. پس کوچک‌ترین زیر‌فضا‌ی شامل همه‌ی ماتریس‌های متقارن و پایین‌مثلثی، همان $M_{n}(R)$ است.\\\\
\textbf{مثال:}
فرض کنید $A\in M_{n}(R)$ باشد؛ آن‌گاه، زیر‌مجموعه‌ی $W=\{x\in R^n| Ax = 0\}\subseteq R^n$ یک زیر‌فضا‌ی خطی است که آن را \textbf{فضای پوچ} $A$ می‌نامیم و با نماد $N(A)$ نمایش می‌دهیم.

\textbf{گزاره: }
فرض کنید $A\in M_{n}(R)$ باشد؛ در این صورت،

۱. A وارون‌پذیر است اگر و تنها اگر $N(A) = \{ 0 \}$

۲. A وارون‌پذیر است اگر و تنها اگر $C(A) = R^n$

\textbf{برهان: }

۱. برای اثبات جهت اول، فرض می‌کنیم $Ax = 0$؛ نتیجه می‌گیریم $x = A^{-1} 0 = 0$ و در نتیجه $N(A) = \{ 0 \}$. برای اثبات جهت دوم، تجزیه‌ی LU برای A را در نظر می‌گیریم. فرض می‌کنیم $LUx = 0$؛ ماتریس L نماینده‌ی اعمال سطری و در نتیجه وارون‌پذیر است، پس می‌توانیم نتیجه بگیریم که $Ux = 0$. ادعا می‌کنیم که عناصر روی قطر اصلی U ناصفر هستند.

$$Ux = \begin{bmatrix}
u_{11} &&*\\
&\ddots&\\
0 && u_{nn}
\end{bmatrix}x = 0$$

به ازای هر i از مجموعه‌ی $\{1,\cdots ,n\}$، داریم $u_{ii}=0$؛ چرا که در غیر این صورت، با تبدیل U به ماتریس تحویل‌یافته‌ی سطری پلکانی $U'$ خواهیم دید که دستگاه $U'x=0$ و در نتیجه $Ux=0$ متغیر آزاد دارد و در نتیجه،
$$N(A) = N(U) \neq 0$$

حال ثابت می‌کنیم هر ماتریس بالامثلثی با درایه‌های قطر اصلی ناصفر، وارون‌پذیر است. با استقرا روی n ثابت می‌کنیم. به ازای شرط پایه‌ی $n=2$،
$$U = \begin{bmatrix}
u_{1} && a\\
0 && u_{2}
\end{bmatrix}$$
و وارون آن،
$$\begin{bmatrix}
\frac{1}{u_{1}} && \frac{-a}{u_{1} u_2}\\
0 && \frac{1}{u_{2}}
\end{bmatrix}$$
است.

فرض استقرا: فرض کنید هر ماتریس بالامثلثی $U_{n-1}$ در $M_{n-1}(R)$ با درایه‌های قطر اصلی ناصفر، وارون‌پذیر است.

حکم استقرا: برای  $U_{n}$ در $M_{n}(R)$ با شرایط مفروض، وارون‌پذیری را ثابت می‌کنیم. قرار دهید:
$$U_n = \begin{bmatrix}
a && C^T\\
0 && U_{n-1}
\end{bmatrix}$$
وارون $U_n$، ماتریس 
$${U_n}^{-1} = \begin{bmatrix}
\frac{1}{a} && x^T\\
0 && {U_{n-1}}^{-1}
\end{bmatrix}$$

است که در آن،

$$x^T = - \frac{1}{a} C^T {U_{n-1}}^{-1}$$

درنتیجه اگر $N(A) = \{ 0 \}$ آن‌گاه در $A = LU$، ماتریس U وارون‌پذیر است، پس داریم

$$A^{-1} = U^{-1} L^{-1}$$


۲. برای اثبات جهت اول، فرض می‌کنیم A وارون‌پذیر باشد. ستون‌های A را با $A_i$ ها نمایش می‌دهیم. آن‌گاه،
$$C(A) = \{ c_1 A_1 + \cdots + c_n A_n | c_i \in R, 1 \leq i \leq n \}$$
زیرمجموعه‌ای از $R^n$ است. فرض کنید b عضوی از $R^n$ باشد. باید $c_i$ های حقیقی بیابیم که $c_1 A_1 + \cdots + c_n A_n = b$. از طرفی،
$$c_1 A_1 + \cdots + c_n A_n = \begin{bmatrix}
A_1 && \cdots && A_n
\end{bmatrix} \begin{bmatrix}
c_1 \\
\vdots \\
c_n
\end{bmatrix} = A \begin{bmatrix}
c_1 \\
\vdots \\
c_n
\end{bmatrix} = b$$
حال، چون A وارون‌پذیر است،
$\begin{bmatrix}
c_1 \\
\vdots \\
c_n
\end{bmatrix} = A^{-1} b$ و در نتیجه،
$C(A) = R^n$.

برای اثبات جهت دوم، می‌دانیم $C(A) = R^n$، در نتیجه به ازای هر b که در $R^n$ است، بردار $c \in R^n$ وجود دارد که $Ac=b$ باشد؛ بنابراین به ازای هر i که در نظر بگیریم، $Ax=e_i$ جواب دارد. فرض کنید $c_i$ جواب $Ax=e_i$ باشد و قرار دهید $c = \begin{bmatrix}
c_1 \\
\vdots \\
c_n
\end{bmatrix}$.
در این صورت، $AC=I$ و اصطلاحا می‌گوییم A وارون راست دارد. حال ثابت می‌کنیم که C و در نتیجه A وارون‌پذیرند. برای اثبات، از مورد ۱ استفاده می‌کنیم و نشان می‌دهیم $N(C) = \{ 0 \}$. فرض کنید $Cx=0$؛ در نتیجه $ACx = Ix = x = 0$ و پس x صفر است، یعنی $N(C) = \{ 0 \}$ و در نتیجه C وارون‌پذیر است، و $AC=I$ پس $CA=I$ و A نیز وارون‌پذیر است.



