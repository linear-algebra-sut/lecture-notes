
\textbf{بسط دترمینان نسبت به ستون}\\
\textbf{گزاره:} فرض کنید $A$ ماتریسی $n\times n$ است، در این صورت:
$$det\: A = \sum_{i=1}^n (-1)^{i+j}a_{ij}det\: A(i|j)$$
\textbf{برهان:} تابع $\Phi: R^n\time\cdot\times R^n $ با ضابطه‌ی $\Phi (a_1,\cdots, a_n) = \sum_{i=1}^n (-1)^{i+j}a_{ij}det\: A(i|j)$ را در نظر بگیرید. تابع $\Phi$ ، $n$ - خطی و متناوب است. لذا بنا به قضیه‌ی رده‌بندی:
$$\Phi(a_1,\cdots,a_n)= (det\: A)\Phi(e_1,\cdots,e_n)$$
از طرفی $\Phi(e_1,\cdots,e_n)=1$، لذا حکم ثابت می‌شود.\\
\textbf{اثبات شهودی:}
$$det\: A = \sum_{\sigma\in S_n} sgn(\sigma) a_{1\sigma(1)}\cdots a_{n\sigma(n)}$$
حال جملات ظاهر‌شده در در جمع فوق را بر حسب درایه‌های ستون $j$ام مرتب می‌کنیم:
$$det\: A = \beta_{i1}a_{i1} + \cdots + \beta_{ij}a_{ij} + \cdots + \beta_{in}a_{in}$$
و نشان می‌دهیم که $\beta_{ij} = (-1)^{i+j}det\: A(i|j)$. در رابطه‌ی اول در بخش اثبات شهودی $\beta_{ij}a_{ij}$ جمع روی همه‌ی قطر پراکنده‌هایی است که از سطر $i$ام $a_{ij}$ انتخاب شده است. به عبارت دیگر $\sigma_i = j$ به ازای $\sigma\in S_n$. بقیه‌ی عناصر قطر پراکنده نیز از درایه‌های ماتریس $A(i|j)$ انتخاب می‌شود. با توجه به اینکه $det\: A = det\: A^T$ و خاصیت متناوب بودن دترمینان، با جا‌به‌جایی $a_{ij}$ و انتقال آن به $a_{11}$، دترمینان در $(-1)^{(i-1)(j-1)}$ ضرب خواهد شد. زیرا $j-1$ و $i-1$ جا‌به‌جایی ستونی و سطری لازم است. پس:
$$\beta_{ij} = (-1)^{i+j}det\: A(i|j)$$
\textbf{تعریف:}
قرار دهید $c_{ij} = (-1)^{i+j}det\: A(i|j)$ و $c_{ij}$ را همساز $ij$ام ماتریس $A$ گویند.\\
\textbf{یادداشت:}
بنا به تعریف:
$$det\:A = \sum_{i=1}^n a_{ij}c_{ij}$$
به عبارت دیگر اگر قرار دهید $C= \begin{bmatrix}
c_{11}&\cdots&c_{ij}&\cdots&c_{1n}\\
\vdots &&\vdots&&\vdots\\
c_{n1} &\cdots &c_{nj}&\cdots&c_{nn}
\end{bmatrix}$ در این صورت $det\: A$ ضرب داخلی ستون $j$ام $A$ در ستون $j$ام $C$ است.


\textbf{یادداشت:} ادعا می‌کنیم اگر $k\neq j$ آنگاه $\sum_{i=1}^n a_{ik}c_{ij} = 0$.\\
برای اثبات، ماتریس $B$ را چنان بسازید که $B$
همان ماتریس $A$ است که ستون $j$ام آن، ستون $k$ام $A$ تکرار شده است. بنابراین $det\:B = 0$. از طرفی $det\:B = \sum_{i=1}^n a_{ik}c_{ij}$. بنابراین $\sum_{i=1}^n a_{ik}c_{ij} = 0$ به ازای $j\neq k$. ماتریس $B$ به ازای $k<j$:
$$B = \begin{bmatrix}
a_{11}&\cdots&a_{1k}&\cdots & a_{1k} & \cdots &a_{1n}\\
\vdots && \vdots &&\vdots && \vdots\\
a_{n1} & \cdots & a_{nk} & \cdots & \underbrace{a_{nk}}_{\text{ام}j\text{ستون}} & \cdots & a_{nn}
\end{bmatrix}$$
در نتیجه:
$$C^TA = \begin{bmatrix}
c_{11}&\cdots&c_{n1}\\
c_{12}& \cdots & c_{n2}\\
\vdots && \vdots\\
c_{1n}& \cdots & c_{nn}
\end{bmatrix}\begin{bmatrix}
a_{11}&\cdots&a_{n1}\\
a_{21}& \cdots & a_{2n}\\
\vdots && \vdots\\
a_{n1}& \cdots & a_{nn}
\end{bmatrix} = \begin{bmatrix}
det\: A && 0 \\
& \ddots &\\
0 && det\: A
\end{bmatrix} = (det\:A )I  $$
\textbf{تعریف:} ماتریس $C^T$ را ماتریس الحاقی کلاسیک ماتریس $A$ گویند و با نماد $adj\: A$ نمایش می‌دهند. پس $(adj\: A)A = (det\:A)I$.

\textbf{خواص:}
\begin{itemize}
	\item $(adj\: A)^T = adj\: A^T$. زیرا:
	$$(adj\: A^T)_{ij} = (-1)^{i+j} det\:A^T(j|i) = (-1)^{i+j}det\: A(i|j) = (adj\: A)_{ji}= (adj\: A)_{ij}^T$$
	\item $(adj\: A)A = A(adj\: A)$. زیرا از رابطه‌ی ${(adj\:A)A = (det\:A)I}$ داریم: $${A^T(adj\:A)^T = (det\:A) I}$$
	پس:
	$$A^Tadj\:A^T = (det\:A)I$$
	از طرفی 
	$$(adj\: A^T) = (det\: A)I$$
	پس:
	$$(adj\: A^T)A^T = A^T(adj\: A^T)$$
	کافی است قرار دهید $B = A^T$ و به دست می‌آید:
	$$(adj\:A^T) = A(adj\: A)$$
	\item اگر $A$ وارون‌پذیر باشد آنگاه:
	$$A(\frac{adj\: A}{det\: A}) = (\frac{adj\:A}{det\: A})A = I \quad \Rightarrow \quad A^{-1} = (\frac{adj\:A}{det\: A})$$
\end{itemize}
\textbf{قاعده‌ی کرامر برای حل دستگاه خطی $Ax = b$}\\
فرض کنید $A\in M_n(R)$ ماتریس وارون‌پذیر باشد. می‌خواهیم دستگاه $Ax=b$ را حل کنیم.
$$Ax = b \quad \Rightarrow \quad (adj\:A)Ax = (adj\:A)b \quad\Rightarrow\quad (det\:A)x = (adj\: A)b$$
$$\Rightarrow\quad (det\:A)x_j = \sum_{i=1}^n(-1)^{i+j}det\:(A(i|j))b_i$$
بنابراین اگر $B_j$ ماتریسی $n\times n$ باشد که از قرار دادن $b$ به جای ستون $j$ام ماتریس $A$ به دست آمده باشد:
$$B_j = \begin{bmatrix}
a_{11} &\cdots&b_1 & \cdots& a_{1n}\\
\vdots && \vdots&& \vdots\\
a_{n1} & \cdots & b_n & \cdots & a_{nn}
\end{bmatrix}$$
در صورت بسط دترمینان $B_j$ نسبت به ستون $j$ام داریم:
$$det\: B_j = \sum_{i=1}^n (-1)^{i+j}det\:(A(i|j)b_i)$$
در نتیجه به ازای هر $j$، $1\leq j \leq n$، داریم:
$$x_j = \frac{det\: B_j}{det\: A}$$
که به عبارت بالا، قاعده‌ی کرامر گویند.\\
\textbf{یادداشت:} فرض کنید $V$ یک فضا‌ی خطی با بعد متناهی و $T: V \rightarrow V$ تبدیل خطی است. مجموعه‌ی $B$ را پایه‌ای برای $V$ در نظر بگیرید. در این صورت دترمینان $T$ را با نماد $det\:T$ نمایش می‌دهیم و تعریف می‌کنیم:
$$det\: T = det\: [T]_B$$
توجه کنید که $det\:T$ مستقل از پایه‌ی $B$ است. زیرا فرض کنید $B^\prime$ پایه‌ی دیگری برای $V$ باشد، در این صورت ماتریس وارون‌پذیر $P$ وجود دارد به طوری که $[T]_B = P[T]_BP^{-1}$. در نتیجه:
$$det\: [T]_B = (det\: P)(det\: [T]_{B^\prime})(det\: P^{-1})$$
پس:
$$det\: [T]_B = det\: [T]_{B^\prime}$$

\chapter{بردارهای ویژه و مقادیر ویژه}

\textbf{مقدمه}

فرض کنید دستگاه معادلات دیفرانسیل زیر با مقادیر اولیه‌ی $u_1(0)=8$ و $u_2(0)=5$ داده شده است.

$$ \frac{du_1(t)}{dt} = 4u_1(t)-5u_2(t) $$
$$ \frac{du_2(t)}{dt} = 2u_1(t)-3u_2(t) $$

به یاد دارید که جواب معادله‌ی
$ \frac{du}{dt} = au(t) $
به ازای $u(0)=u_0$ برابر با $u(t)=e^{at}u(0)$ است. با پیروی از جواب معادله‌ی تک‌مجهولی، حدس می‌زنیم که $u(t)=ae^{\lambda t}$ که
$a=\begin{bmatrix}
a_1 \\
a_2
\end{bmatrix}$
باید جواب دستگاه فوق باشد، یعنی

$$ \lambda e^{\lambda t} a_1 = 4 e^{\lambda t} a_1 - 5 e^{\lambda t} a_2 $$
$$ \lambda e^{\lambda t} a_2 = 2 e^{\lambda t} a_1 - 3 e^{\lambda t} a_2 $$

بنابراین بایستی $a$ و $\lambda$ را به گونه‌ای بیابیم که

$$ \lambda  a_1 = 4  a_1 - 5 a_2 $$
$$ \lambda a_2 = 2  a_1 - 3 a_2 $$

و در نتیجه

$$ \begin{bmatrix}
4 & -5 \\
2 & -3
\end{bmatrix}
\begin{bmatrix}
a_1 \\
a_2
\end{bmatrix}
=
A
\begin{bmatrix}
a_1 \\
a_2
\end{bmatrix}
=
\lambda
\begin{bmatrix}
a_1 \\
a_2
\end{bmatrix} $$

به عبارت دیگر، در صورتی‌که بتوان $\lambda$ و بردار $a$ را به‌گونه‌ای یافت که $Aa=\lambda a$، آن‌گاه
$u(t)=ae^{\lambda t}$
جواب دستگاه اولیه خواهد بود. بردار $a$ خاصیت ویژه‌ای دارد: راستای این بردار، تحت $A$ حفظ می‌شود؛ به عبارت دیگر،
$$(A-\lambda I)a=0$$
بنابراین، اگر دنبال بردار ناصفری مانند $a$ هستیم که تحت $A$ راستای آن حفظ شود، باید
$N(A-\lambda I)\neq \{0\}$
و به عبارت دیگر، باید $\lambda$ را به گونه‌ای بیابیم که فضای پوچ
$A-\lambda I$
نابدیهی باشد، یعنی باید
$A-\lambda I$
تکین باشد، معادلا:
$$det(A-\lambda I)=0$$

در مثال فوق،

$$A-\lambda I = \begin{bmatrix}
4-\lambda & -5 \\
2 & -3-\lambda
\end{bmatrix} $$

$$det(A-\lambda I) = (4-\lambda)(-3-\lambda)+10 = \lambda^2 -\lambda - 2 = 0$$

$$ \to \lambda_1 = 2 \; , \; \; \lambda_2 = -1 $$

$$ \lambda = -1 \to (A-\lambda I)a = \begin{bmatrix}
5 & -5 \\
2 & -2
\end{bmatrix}
\begin{bmatrix}
a_1 \\
a_2
\end{bmatrix}
=
\begin{bmatrix}
0 \\
0
\end{bmatrix} \to a = \begin{bmatrix}
a_1 \\
a_2
\end{bmatrix} =
\begin{bmatrix}
1 \\
1
\end{bmatrix} $$
$$ \to u(t) = e^{-t} \begin{bmatrix}
1 \\
1
\end{bmatrix} \; \; (2)$$


$$ \lambda = 2 \to (A-\lambda I)a = \begin{bmatrix}
2 & -5 \\
2 & -5
\end{bmatrix}
\begin{bmatrix}
a_1 \\
a_2
\end{bmatrix}
=
\begin{bmatrix}
0 \\
0
\end{bmatrix} \to a = \begin{bmatrix}
a_1 \\
a_2
\end{bmatrix} =
\begin{bmatrix}
5 \\
2
\end{bmatrix} $$
$$ \to u(t) = e^{2t} \begin{bmatrix}
5 \\
2
\end{bmatrix}  \; \; (1)$$

جواب‌های (۱) و (۲) جواب‌های خاص معادله‌ی دیفرانسیل هستند، بنابراین مجموعه جواب عمومی معادله‌ی دیفرانسیل فوق، عبارت است از:

$$ \{ u(t) = c_1 e^{-t} \begin{bmatrix}
1 \\
1
\end{bmatrix} + c_2 e^{2t} \begin{bmatrix}
5 \\
2
\end{bmatrix} \Biggr| c_1 \in R \; , \; \; c_2 \in R \} $$

پس جواب معادله‌ی دیفرانسیل، تحت شرایط اولیه‌ی داده شده، عبارت است از:

$$ u(t) = 3 e^{-t} \begin{bmatrix}
1 \\
1
\end{bmatrix} + e^{2t} \begin{bmatrix}
5 \\
2
\end{bmatrix} $$

زیرا، با لحاظ کردن شرایط اولیه‌ی $u_1(0)=8$ و $u_2(0)=5$ خواهیم داشت:

$$ \begin{bmatrix}
1 & 5 \\
1 & 2
\end{bmatrix}
\begin{bmatrix}
c_1 \\
c_2
\end{bmatrix} =
\begin{bmatrix}
8 \\
5 
\end{bmatrix} \to c_1 = 3 \; , \; \; c_2 = 1 $$


